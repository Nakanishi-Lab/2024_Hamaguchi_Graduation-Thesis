\newpage
\setcounter{page}{1}
\section{緒言}
McKibben型空気圧人工筋(McKibben Pneumatic Actuator,以下MPA)は圧縮空気を入力することで収縮し,自身の軸方向への張力を発生させるアクチュエータである\cite{2003722}.
従来は直径が数十mm程度のMPAを用いたロボットに関する応用研究が盛んに行われてきたが,近年では直径が数mm程度の細径MPAが注目を集めている\cite{wakimoto}.
細径MPAは従来のものより細くしなやかであり生体筋に似た特徴から小さな筋肉,あるいは集積によって単純な紡錘型以外の筋肉を表現可能なため,筋骨格系ロボットや生物模倣ロボットなどに盛んに用いられてきた\cite{森田隆介2016}\cite{森和也2014}.

一方で甲殻類や昆虫などを模した外骨格生物模倣ロボットに関しては外骨格内部へアクチュエータを配置することが困難なことから,関節にサーボモータを配置したもの\cite{jmse10121804}が主流となっている.
このようなロボットは外骨格生物の外形こそ再現できているものの,実際の生物の構成や駆動原理からして異なる.
また「構成要素が外骨格内にすべて納まっている」という外骨格ならではのメリットも,実現できているとは言い難い.
これに対して前述の細径MPAは,その細さやしなやかさから細長い外骨格内部に配置可能であり,また集積することで実際の生物のような羽状筋も表現することが可能である\cite{2003}.

そこで本研究では細径MPAを用いた外骨格生物模倣ロボットの開発を行う.
先行研究\cite{hasegawa}では外骨格生物のなかでも甲殻類の蟹(ズワイガニ)をモデル生物として設定し,これの歩脚を模倣したロボットの開発に向けて,細径MPAおよびそれを用いた羽状筋の開発が行われた.
またズワイガニを実際に解剖して得られた計測データを参考生歩脚ロボットを開発し,細径空圧羽状筋によって関節を開閉動作させることに成功した\cite{hasegawa}.
一方で,先行研究\cite{hasegawa}で開発された細径空圧羽状筋およびロボットにはいくつかの問題が確認された.
まず,先行研究で用いられた細径MPA作製方法は煩雑であり,作製に時間と練度が必要であった.また収縮性能が低いことや,羽状筋の細径MPAの根元の角度が固定されており,
収縮時に羽状角が変化できずに腱の引き込みを妨げていた.またロボットの関節部についても腱の付着構造に問題があり,可動域の再現などはできていなかった.

そこで本研究では,改めてカニの解剖を行い,関節構造などを確認した.さらに前述した課題点を見直した改良細径空圧筋およびロボットを開発した.
本論文の構成は以下の通りである.まず2章では,本研究で用いる細径MPAに関する特徴と先行研究について述べてから,本研究で開発に成功した細径MPAの作製方法を紹介する.
次に3章では,本研究でモデル生物として扱う蟹の構成と筋構造について実際にズワイガニを解剖して得た知見などを基に述べる.
最後に4章では模倣ロボットを作製するにあたって集積細径MPAを用いた羽状筋の開発について述べたのち,ズワイガニをモデルにした歩脚ロボットの開発とその動作実験について述べる.

% 代表的な人工筋肉として,圧縮空気により骨格筋のように収縮するMcKibben 型空気圧人工筋肉(MPA) があげられる.
% 従来は直径が数十mm 程度のものが多かったが,近年では数mm 程度の細径のMPA が注目を集めている\cite{wakimoto}.
% その細さを生かして小さな筋肉,あるいは集積によって単純な紡錘型以外の筋肉を表現可能なことから,筋骨格系ロボットにおいて特に盛んに用いられている\cite{wakimoto}.
% 一方で甲殻類をはじめとする外骨格系ロボットについては,ワイヤ駆動や関節にサーボモータを配置したものが主流であった\cite{jmse10121804}.
% これは骨格内部にアクチュエータを配置することが困難だからである.
% 細径MPA であれば骨格内部にアクチュエータを配置することが可能であり,実際の生物に近い構成でロボットを作製することが可能である.
% そこで本研究では外骨格生物のうち甲殻類の蟹をモデルに,実際の蟹の筋構造を参考にして細径MPA を使用した蟹の歩脚ロボットの開発に取り組む.
