\newpage
\section{結言}
本研究では,細径MPAを用いて羽のような構造である羽状筋を再現し,それを用いてカニの脚を模したロボットの開発を目的として研究を行った.


本稿ではまず先行研究において確認された細径MPAを用いた羽状筋に関する3つの課題について対策を講じ,改良型細径空圧羽状筋の開発を行った.
細径MPAの作製方法を見直し,Oリングを用いた締結方法を開発することで簡易かつ確実に製作することが可能となった.
またメッシュの後加工手法を開発することで,収縮率の向上に成功した.
さらに細径MPAを羽状配置した際,羽状角が可変となるように端部部品に回転自由度を持たせる改良を行った.
ロボットの設計については,カニの羽状筋および関節構造を数理モデルとして定式化し,所望の関節可動域に対して細径MPAの長さや羽状角がどの程度必要かを検討した.
その後,改良型細径空圧羽状筋と外骨格を用いて歩脚ロボットを作製し,動作実験を行った.
動作実験では長節から腕節間,腕節から前節間の開閉動作を確認することが出来た.
またトラッキングソフト(kinovia)を用いて可動域の計算を行った結果,全ての関節において多少の誤差はあるものの,概ね設計通りの可動域が実現されていることが確認できた.


今後は多様な長さの細径MPAを配置する方法,細径MPAに圧縮空気を印加していない状態で張力が発生しない固定方法を考えることにより実際の蟹の動きに近づけるような歩脚ロボットの開発を予定している.